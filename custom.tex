\usepackage{fontspec} % needed for LuaLaTeX/XeLaTeX
% Choose an emoji font you have; Noto Color Emoji is common
\newfontfamily\emojifont{Noto Color Emoji}[Renderer=Harfbuzz]
\newcommand{\emoji}[1]{{\emojifont #1}}

% Colors
\usepackage{xcolor}
\definecolor{erasmusblue}{HTML}{00A1E4}
\definecolor{erasmuslightblue}{HTML}{E6F5FB}
\definecolor{advancedorange}{HTML}{FF9800}
\definecolor{advancedlightorange}{HTML}{FFF4E6}

\usepackage[most]{tcolorbox}
\tcbset{
  enhanced,
  breakable,
  sharp corners,
  boxsep=0pt,
  left=28pt,   % room for icon + gutter
  right=10pt,
  top=10pt,
  bottom=10pt,
  leftrule=5pt,
  toprule=0pt,
  bottomrule=0pt,
  rightrule=0pt,
}

% A small helper to place an icon at top-left for all split segments
\newcommand{\CalloutIconOverlays}[1]{%
  overlay unbroken={\node[anchor=north west] at (frame.north west) {\raisebox{.2ex}{\large \emoji{#1}}};},
  overlay first={\node[anchor=north west] at (frame.north west) {\raisebox{.2ex}{\large \emoji{#1}}};},
  overlay middle={\node[anchor=north west] at (frame.north west) {\raisebox{.2ex}{\large \emoji{#1}}};},
  overlay last={\node[anchor=north west] at (frame.north west) {\raisebox{.2ex}{\large \emoji{#1}}};},
}

% Redefine callout environments for custom types
\newenvironment{callout-background}
{%
  \begin{tcolorbox}[
    colback=erasmuslightblue,
    colframe=erasmusblue,
    \CalloutIconOverlays{📚}
  ]
}
{%
  \end{tcolorbox}
}

% Advanced callout
\newenvironment{callout-advanced}
{%
  \begin{tcolorbox}[
    colback=advancedlightorange,
    colframe=advancedorange,
    \CalloutIconOverlays{🎓}
  ]
}
{%
  \end{tcolorbox}
}